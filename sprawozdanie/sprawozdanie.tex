\documentclass[a4paper,11pt]{article}

\usepackage[T1]{polski}
\usepackage[utf8]{inputenc} 
\usepackage{graphicx}
\usepackage{float}
\usepackage{verbatim}
\usepackage[obeyspaces]{url}
\usepackage{hyperref}
\usepackage{xcolor}
\usepackage{lastpage}
\usepackage{fancyhdr}
\usepackage{enumitem}
\usepackage{titling}
\usepackage{adjustbox}
\usepackage{array}
\usepackage{changepage}
\usepackage{booktabs}

\newcolumntype{R}[2]{%
	>{\adjustbox{angle=#1,lap=\width-(#2)}\bgroup}%
	l%
	<{\egroup}%
}
\newcommand*\rot{\multicolumn{1}{R{90}{1em}}}
\renewcommand\maketitlehooka{\null\mbox{}\vfill}
\renewcommand\maketitlehookd{\vfill\null}

\pagestyle{fancy} 
\fancyhf{}
\renewcommand{\headrulewidth}{0pt} 

\cfoot{\thepage \hspace{1pt} z \pageref*{LastPage}}
\rfoot{\hyperlink{toc}{Spis treści}}

\hypersetup{
    colorlinks,
    linkcolor={black},
    filecolor={blue}
}

\newcommand{\linkToFile}[2]{\href[pdfnewwindow=true]{#1}{\nolinkurl{#2}}}

\title{ Symulator inteligentnego sterowania windą \\ Sprawozdanie}
\author{Artur Gajowniczek \\ Michał Kośmider \\ Michał Kalisiak \\ Karol Mąkosa \\ Rafał Woźniak}

\begin{document}
\begin{titlingpage}
\maketitle
\end{titlingpage}
\thispagestyle{empty}
\cleardoublepage

\setcounter{page}{1}

\hypertarget{toc}{}
\tableofcontents

\cleardoublepage

\section{Wprowadzenie}
Systemy wbudowane i czasu rzeczywistego znajdują w dzisiejszym świecie coraz więcej zastosowań. Nowoczesna technologia przyzwyczaiła ludzi do bardzo krótkiego czasu odpowiedzi w niemalże wszystkich systemach komputerowych. Powstało też wiele systemów, w których z góry określony czas odpowiedzi jest kluczowy dla ich poprawnego działania i to właśnie one są nazywane systemami czasu rzeczywistego. Systemy wbudowane natomiast są nierozerwalnie połączone z urządzeniem, które obsługują. Jednym z przykładów systemu, który jest jednocześnie systemem wbudowanym oraz czasu rzeczywistego, jest system inteligentnego sterowania windą.
\\
\\
Problem, który rozwiązuje system sterowania windą jest w dużym stopniu podobny do dobrze znanego Problemu Komiwojażera, jednak występuje kilka znaczących różnic między nimi. Główną różnicą jest fakt, że w Problemie Komiwojażera całe zadanie jest od początku znane, natomiast w naturalnym środowisku windy warunki zadania cały czas się zmieniają, ponieważ pojawiają się nowi ludzie. Z tego powodu nie można z góry wyliczyć optymalnego rozwiązania, algorytm sterowania musi brać pod uwagę zmiany przy podejmowaniu kolejnych decyzji.

\section{Istniejące rozwiązania}
Inteligentne sterowanie windą nie jest nowym problemem i został już opracowany niejeden algorytm z nim związany. Większość z nich zajmuje się sterowaniem grupą wind, a nie pojedynczą windą. Poniżej znajduję się lista kilku z tych algorytmów oraz strategii wraz z krótkimi opisami.

\begin{itemize}
\item Strategia kontroli kolektywnej [1] -- w tym algorytmie winda porusza się w jednym kierunku zabierając tylko pasażerów jadących w tym samym kierunku. Gdy wszyscy pasażerowie wysiądą i nie ma następnych pasażerów znajdujących się dalej w danym kierunku niż winda, winda zmienia kierunek jeśli są jakieś inne wezwania. W przeciwnym wypadku winda będzie czekać na piętrze na którym wysiadł ostatni pasażer.
\item Strategie przeszukujące [1] -- w odróżnieniu od poprzedniej strategii, która należy do rodziny algorytmów zachłannych, strategie przeszukujące będą analizować wiele (lub wszystkie) możliwości i ich konsekwencje, a następnie podejmować decyzje wybierając spośród możliwości tę, która najlepiej spełnia zadane kryterium optymalizacyjne np. 
\end{itemize}

\section{Bibliografia}
[1] https://www.diva-portal.org/smash/get/diva2:668654/FULLTEXT01.pdf

\end{document}